\chapter{Schnellstart}
Wer sich mit dem IRC auskennt und schon die Eigenheiten einiger IRC-Programme
kennen gelernt hat, m�chte nicht unbedingt die ganze Dokumentation w�lzen um an
die jeweiligen Einstellungen des XChats zu kommen. Darum f�r all jene, die
diese sch�ne Dokumentation missen m�chten, sei hier ein Schnellstart
dokumentiert ;)
Als Beispiel nehme ich den \texttt{irc.euirc.net} Server und als Kanal
\texttt{\#studies}.

\section{XChat f�r Linux/Unix}

Ich gehe davon aus, dass die Paketverwaltung der jeweiligen Distribution das
XChat Paket auf den Rechner gebracht hat. Sollte das nicht der Fall sein, wird
man auf Seite \pageref{bekommen_starten} unter (\ref{bekommen_starten}) eher
f�ndig werden.  Gestartet wird der XChat durch das Kommando
\verb|xchat|\footnote{In manchen Distributionen wie z.B. Debian hei�t die
  ausf�hrbare Datei \texttt{xchat-gnome}. Abh�ngig ist dies vom Hersteller des Paketes
  f�r die Distribution.} oder \verb|xchat-gnome|.

\section{XChat f�r Windows}
Mit dem Browser geht man auf \texttt{http://www.xchat.org} und l�dt sich
die neuste Windows Version des XChat herunter.
Nach dem Download des Pakets, l�sst sich der XChat wie ein
normales Windowsprogramm �ber einen Installations-Wizard installieren. Zudem
sollten im Programmstartmen� nach der erfolgreichen Installation die
Men�eintr�ge enthalten sein, oder je nach Wunsch auf dem Desktop ein Icon.
�ber das Startmen� l�sst sich das Programm starten.

\section{Verbindung aufbauen und chaten}
Nach dem Start, sieht man das Serverfenster welches schon vorkonfiguriert einige
Netzwerke beinhaltet. Da es sich hier um einen Schnellstart handelt, soll uns diese
Liste nicht weiter interessieren, also weg klicken wir sie weg. Im Hauptfenster
verbindet man sich mit folgenden Kommandos zum Server:
\begin{quote}
  \texttt{/server irc.euirc.net}
\end{quote}

Nachdem absetzen des Kommandos sollte nach der Verbindung einiges an Text
in dem Textfenster zu sehen sein (Message of the Day, Verbindungs- und Benutzerstatistiken). Daraufhin kann 
man einen Kanal beitreten:
\begin{quote}
  \texttt{/join \#studies}
\end{quote}

Nebenbei kann man jetzt in aller Ruhe die Eigenheiten und Funktionen des XChats
kennen lernen, indem man diese Dokumentation liest.


