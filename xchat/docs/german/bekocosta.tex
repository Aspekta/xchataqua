
\chapter{Bekommen, Compilieren und Starten}\label{bekommen_starten}
\section{Was ist XChat?}

XChat ist ein grafischer IRC Client, welcher unter Unix �hnlichen Systemen l�uft. Es benutzt das GTK+ Toolkit f�r die grafische Oberfl�che. Es ist GPLed Software (Freie Software). 
Unter folgenden Systemen sollte es laufen:

\begin{itemize}
\item Linux (prim�re Entwicklungsplattform)
\item FreeBSD
\item OpenBSD
\item NetBSD
\item Solaris
\item AIX
\item IRIX
\item SunOS
\item OS/2
\item MS Windows
\end{itemize}


\section{Bekommen}
In den meisten Linux Distributionen ist das XChat Paket schon enthalten und
kann �ber die verschiedenen Installationsroutinen der Distribution installiert
werden. F�r andere Plattformen, wie zum Beispiel Windows gibt es die XChat
Homepage\footnote{http://www.xchat.org}. Hier kann man die Windows Bin�rpakete
runterladen, die sich wie ein normales Windows Programm installieren lassen.
Zus�tzlich gibt es noch weitere Wege ``inoffizielle'' Windows Pakete von XChat
zu bekommen. Die Seite: \verb|http://ursus.mif.vu.lt/xchat/news/| bietet einen
guten Anlaufpunkt f�r aktuelle Windows-Versionen. Angemerkt sei aber hier,
dass die meisten Seiten in englischer Sprache sind.

\section{Compilieren unter Unix/Linux}
\textbf{Hinweis:} Wer sich nicht mit dem Entwickeln von Programmen unter
Unix/Linux auskennt, sollte die oben genannten Distributionswege ausw�hlen.
Das Compilieren der Quelldateien setzt Programmierwissen vorraus.

XChat benutzt das ``GNU autoconf system'', so dass das Compilieren sehr leicht
sein sollte. F�r die meisten Systeme wird die automatische Erkennung
funktionieren: \texttt{./configure ; make ; su ; make install}. Auf einigen
Systemen wird \verb|gmake| mehr gebraucht, als \verb|make|. Dem
Konfigurationsscript (\verb|configure|) kann man noch einige Optionen �bergeben:
\begin{itemize} \item \verb|--disable-perl| = Schaltet die PERL Unterst�tzung aus
  \item \verb|--disable-gnome| = Schaltet die GNOME Unterst�tzung aus \end{itemize}

Es sei darauf hingewiesen, dass das Script diese Optionen automatisch setzt,
wenn kein GNOME oder PERL installiert ist. Sie sind nur f�r den Fall gedacht,
wenn man GNOME oder PERL installiert hat, aber es nicht nutzen m�chte. Eine
�bersicht �ber die m�glichen Optionen die \verb|configure| ben�tigt, kann man
sich mit 
\begin{verbatim}
  ./configure --help
\end{verbatim} 
ausgeben lassen.

\section{Starten}

Die Compilierung erzeugt eine Bin�rdatei namens \verb|xchat|. Wenn Du es installiert
hast, kannst du das Programm mit \verb|xchat| starten.
Ansonsten einfach in das XChat Verzeichnis und \verb|./xchat| eintippen.


Dein pers�nliches Verzeichnis \verb|~/.xchat| sollte f�r Dich automatisch erstellt werden.
XChat benutzt das Verzeichnis, um benutzerspezifische Einstellungen und Logs
abzulegen. 
